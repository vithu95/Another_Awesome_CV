% Awesome Source CV LaTeX Template
%
% This template has been downloaded from:
% https://github.com/darwiin/awesome-neue-latex-cv
%
% Author:
% Christophe Roger
%
% Template license:
% CC BY-SA 4.0 (https://creativecommons.org/licenses/by-sa/4.0/)

%Section: Project
\sectionTitle{Konferencer / Frivillige aktiviteter}{\faPlus}

\begin{projects}
\project
{Repair Cafe Aarhus}{}
{\website{https://www.facebook.com/RepairCafeAarhus/}{Repair Cafe Aarhus}}
{Frivillig repairmand (cykler og computer - 5 år) fra 2016, formand fra 2022 (2 år)
\begin{itemize}
    	\item Planlægning af arrangementer med \website{https://www.dokk1.dk}{Dokk1}
        \item Rekruttering af frivillige
        \end{itemize}}
{repair, reuse, recycle, bæredygtighed, planlægning}
%\emptySeparator

\project
{Aarhus Bueskyttelaug}{}
{\website{https://www.aarhusbueskyttelaug.dk}{Aarhus Bueskyttelaug}}
{Frivillig bueskydningstræner, holder arrangementer til både voksne og børn}
{}

\project
{Det grønne Akademi | grøn omstilling i Aarhus og Danmark | 2023}{}
{\website{https://dga.aarhus.dk}{Det grønne akademi}}
{Seminaret satte fokus på de forskellige aspekter af bærdygtighed
\begin{itemize}
    	\item Økonomi, virksomheds modeller (cirkulær-/deleøkonomi,leasing aftaler, osv).
        \item Livs cyklus analyse af produkter, CO2 beregning
        \item Virksomheds processer
        \item Social bæredygtighed
        \end{itemize}
}
{Bæredygtighed, Byplanlægning, Langsigtet planlægning}
%\emptySeparator


\project
{Vinterakademi | Den bæredygtig og robuste by | 2017}{}
{\website{https://www.22decembre.eu/en/2017/11/25/vinterakademi-2017/}{Vinterakademi 2017}}
{Seminaret satte fokus på de forskellige roller hvor folk (borgere, fagfolk, politikere) skal bidrage i en bæredygtig byudvikling.}
{}
%{bæredygtighed, byplanlægning, langsigtet planlægning}
%\emptySeparator

\project
	{Vinterakademi | Cirkularøkonomien i bygningsområdet | 2016}{}
	{\website{http://www.vinterakademi.dk}{Vinterakademi} \\
    \website{https://www.22decembre.eu/da/2016/08/04/circular/}{Cirkularøkonomien i bygningsområdet … eller cirklens kvadratur på byggeri ?} }
	{ Seminaret gav stor input omkring bæredygtighedsprocessen i bygning og renovering, fra opsætning af bygninger, til nedtagning, med fokus på reuse og cirkulær tænkegang.
 \begin{itemize}
    	\item Designmetoder til bæredygtighed
        \item Deltageres roller i bæredygtighedsprocessen og materialer.
        \item Case arbejde omkring en byging på Refshaleøen, i København.
        \end{itemize}
        }
	{repair, reuse, recycle, bæredygtighed, planlægning}
\end{projects}